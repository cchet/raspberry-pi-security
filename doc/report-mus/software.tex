
\section{Software}
Dieser Abschnitt behandelt die verwendete bzw. implementierte Software für \emph{RPISec}.
\subsection{\emph{Microservices} und \emph{Cloud}}
Dieser Abschnitt behandelt die auf dem \emph{Raspberry PI} gehosteten Services. Die Services wurden mit \emph{Spring Boot} als \emph{Microservices} implementiert, was möglich war, da Oracle eine ARM Implementierung der Java-JDK bereitstellt und die \emph{Microservices} schlank implementiert wurden, sodass die zur Verfügung stehenden Ressourcen ausreichen, um diese Services auf einen \emph{Raspberry PI} zu betreiben.
\newline
\newline
Es wurden die beiden \emph{Microservices} \emph{rpisec-auth-service} für die Benutzerverwaltung und OAuth2 Authentifizierung und \emph{rpisec-app-service} für die Interaktion mit der Sensorik und der Interaktion mit dem \emph{Cloud}-Diensten implementiert, wobei der \emph{Microservice} \emph{rpisec-auth-service} im Zuge des Projekts für die Lehrveranstaltung \emph{Service Engineering} implementiert wurde. Es hätte auch ausgereicht die Benutzerverwaltung in den \emph{Microservice rpisec-app-service} zu verpacken, obwohl dann der \emph{Microservice} für zwei Aspekte verantwortlich gewesen wäre was im Widerspruch zu einem \emph{Microservice} steht, der nur für einen Aspekt verantwortlich sein soll. 
\newline
\newline
Der Microservice \emph{rpisec-app-service} interagiert nicht direkt mit der Sensorik, sondern bindet die Sensorapplikation beschrieben in Abschnitt \ref{sec:sensor-application} ein und ist für dessen Lebenszyklus verantwortlich. Nachdem Start der Sensorapplikation wird ein \emph{Listener} registriert, der auf Statusänderungen des Bewegungssensor reagiert und diesen Sicherheitsvorfall wie in Abbildung \ref{fig:image-sequence-incident} behandelt.
\newline
\newline
Die beiden \emph{Microservices} müssen Daten persistent halten und sind daher auf eine Datenbank angewiesen, wobei im Entwicklungsbetrieb auf einen Entwicklerrechner H2 und im produktiven Betrieb auf einen \emph{Raspberry PI} PostgreSQL verwendet wird. Die Datenbank PostgreSQL konnte verwendet werden, da PostgreSQL die ARM Architektur unterstützt.
\newline
\newline
Als \emph{Cloud} Anbieter wurde \emph{Google} gewählt, welcher die Plattform \emph{Firebase} anbietet, die eine JSON-Datenbank und einen \emph{Cloud Messaging} Dienst anbietet. Für diesen Dienst gibt es eine Java Implementierung das sogenannte \emph{firebase-admin-sdk}, das eine API zum Interagieren mit der JSON-Datenbank und eine API zum Erstellen von Authentifizierungstoken für die \emph{Client}-Authentifizierung bei Firebase zur Verfügung stellt. In der Java Implementierung wird zurzeit keine API für die Interaktion mit dem \emph{Messaging} Dienst zur Verfügung gestellt, was aber kein Problem darstellt, da es sich hierbei um eine einfache Anfrage an eine \emph{REST-API} handelt, die mit Spring \emph{RestTemplate} durchgeführt wird.