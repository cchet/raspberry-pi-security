\section{Docker Infrastruktur}
Dieser Abschnitt behandelt die Docker Infrastruktur, welche die Service und deren Abhängigkeiten hosted und verwaltet. Da der Umgang mit Docker und einer umfangreicheren Infrastruktur mit viel Shell-Skripten verbunden ist, wird das Python basierte Tool Docker-Compose verwendet, das es erlaubt eine Infrastruktur, die aus einer Menge von untereinander abhängigen Services besteht, deklarativ über eine \emph{YAML}-Konfigurationsdatei definiert werden kann. 
\newline
\newline
Der Quelltext \ref{src:test-docker-compose} zeigt den Inhalt der \emph{docker-compose.yml}, welche die Docker Infrastruktur für \emph{RPISec} amd \emph{Raspberry PI} definiert. Die in der Datei vorkommenden Textfragmente im Format \emph{\$\{...\}} stellen Variablen dar, die \emph{Docker Compose} entweder aus einer Datei mit dem Namen \emph{.env}, die auf derselben Ebene wie die \emph{docker-compose.yml} platziert werden muss, oder aus den Umgebungsvariablen des Benutzers, mit dem die Infrastruktur erstellt wird, auflöst. Sollten Variablen nicht auflösbar sein, so wird eine entsprechende Meldung auf die Konsole ausgegeben.
\begin{code}
	\caption{docker-compose.yml für RPISec am \emph{Raspberry PI}}
	\yamlFile{\dockerRPIDir/docker-compose.yml}
	\label{src:test-docker-compose}
\end{code}