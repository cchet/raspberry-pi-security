\section{Erfahrungen}
Dieser Abschnitt behandelt die gemachten Erfahrungen während der Umsetzung dieses Projekts. Nachdem für dem \emph{Raspberry PI} bereits ein Betriebssystem zur Verfügung steht, das alle benötigten Bibliotheken für wie \emph{Docker} und \emph{Docker-Compose} bereitstellt und es auch eine \emph{ARM} Implementierung von \emph{Oracle}-Java gibt, hat sich die Umsetzung als relativ einfach gestaltet. Hätte es keine Oracle Implementierung für die \emph{ARM}-Plattform gegeben, hätte man \emph{OpenJDK} verwenden müssen, was das Laufzeitverhalten der \emph{Microservices} negativ beeinflusst hätte.   
\newline
\newline
Die verwendete \emph{PostgreSQL} Datenbank, die in einem \emph{Docker Container} gehostet wird, braucht beim ersten Start (Keine Datenbank vorhanden) relativ lange, weswegen der erste Start der \emph{Docker-Compose} orchestrierten  Infrastruktur beim ersten Mal fehlschlägt, da die \emph{Microservices} gestartet sind und auf eine nicht verfügbare Datenbank zugreifen wollen, die noch im Initialisierungsprozess feststeckt. Dieses Problem wurde gelöst indem die Datenbanken initialisiert werden, bevor die gesamte Infrastruktur gestartet wird.
\newline
\newline
Die Java Bibliothek für den \emph{Cloud}-Dienst \emph{Firebase} ist zwar eingeschränkt verglichen mit der Implementierung für \emph{NodeJS}, jedoch konnten alle Tasks, bis auf das Versenden der \emph{Messages}, das mit Spring \emph{Resttemplate} realisiert wurde, einfach implementiert werden. 
\newline
\newline
Für die Umsetzung der \emph{Microservices} mussten keine besonderen Vorkehrungen wegen dem Hosten auf einen \emph{Raspberry PI} getroffen werden. Nichts desto trotz sollte man sich bei der Implementierung der \emph{Microservices} bezüglich der Abhängigkeiten und verwendeten \emph{Frameworks} bewusst sein, dass man mit einem System konfrontiert ist, das nur beschränkte Ressourcen zur Verfügung stellt.
\newline
\newline
Abschließend kann man sagen das \emph{Docker} und Java Programme auf einen \emph{Raspberry PI} sehr gut zu betreiben sind, solange man nicht Programme und \emph{Docker Container} betreiben will, für welche die zur Verfügung stehenden Ressourcen nicht oder nur knapp ausreichen. Vor allem das Zeitverhalten muss berücksichtigt werden, dass aufgrund der \emph{ARM}-Architektur sich nicht gleich verhält als auf einer \emph{x86}-Architektur. \emph{PI4J} bietet einen sehr gute \emph{API} für die Interaktion mit den \emph{GPIO} und für das Ausführen von Prozessen am Hostsystem.  