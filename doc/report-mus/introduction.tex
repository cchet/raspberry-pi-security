\section{Einleitung}
Diese Dokument stellt die Dokumentation des Projekts \emph{Raspberry PI Security}, in weiterer Folge \emph{RPISec} genannt, dar, welches für die Lehrveranstaltung \emph{Mobile und ubiquitäre Systeme} realisiert wurde. In diesem Projekt wird eine Heimsicherheitsanwendung mit einem \emph{Raspberry PI 3 Model B}, \emph{Docker} und \emph{Spring Boot Microservices} umgesetzt, das bei einem Sicherheitsverstoß in der Lage sein soll, bekannte mobile Endgeräte von registrierten Benutzern über diesen Sicherheitsverstoß zu informieren. 

\subsection{Problemdarstellung}
Dieser Abschnitt behandelt die Problemdarstellung, welche die Grundlage für das umzusetzende Projekt \emph{RPISec} ist. Bei einem Auslösen eines an dem \emph{Raspberry PI} angeschlossenen Bewegungssensors,  sollen alle am System registrierten Benutzer über ihre mobilen Endgeräte wie Handy oder Tablet über den Vorfall informiert werden, sowie die Möglichkeit haben ein Bild zu erhalten, welches den Sicherheitsbereich zum Zeitpunkt des Sicherheitsverstoßes zeigt. 
\newline
\newline
Da es sich um eine Sicherheitsanwendung handelt, soll die Benutzerverwaltung sowie die Authentifizierung \emph{In-House} gehalten werden, also die Sicherheitsanwendung selbst soll in der Lage sein die Benutzer zu verwalten und die Authentifizierung der Benutzer durchzuführen. Da sich die mobilen Endgeräte in irgendwelchen Netzen an das Internet anbinden können, wie zum Beispiel über einen Mobilfunkanbieter, Internetanbieter oder öffentlichen \emph{Hot-Spot}, wird ein \emph{Messaging} Dienst benötigt, über den die mobilen Endgeräte erreicht werden können. Dieser \emph{Messaging} Dienst muss es erlauben, dass die Benutzerverwaltung extern erfolgen kann, da wir diesen \emph{Messaging} Dienst nicht vertrauen wollen und daher den \emph{Messaging} Dienst auch nicht die Benutzerverwaltung überlassen wollen. Ebenso wird ein online Speichermedium benötigt, dass alle gemachten Bilder speichert, damit die registrierten Benutzer jederzeit darauf zugreifen können.


\subsection{Funktionsweise}
Dieser Abschnitt behandelt die Funktionsweise von \emph{RPISec}. Die Abbildung \ref{fig:image-system-structure} zeigt den Systemaufbau von \emph{RPISec} mit der involvierten Hardware und \emph{Cloud} Diensten. 
\begin{figure}[h]
	\centering
	\includegraphics[scale=0.7]{\imageDir/Infrastructure.jpg}
	\caption{Systemaufbau der \emph{RPISec} Applikation}
	\label{fig:image-system-structure}
\end{figure}
\ \newpage

\subsubsection{Zugangsverifikation}
Beim Start des Systems wird vom Authentifizierungsservice \emph{(Auth-Service)} ein Administrator Benutzer erstellt, wenn dieser nicht bereits existiert. Ein neu angelegter Benutzer wird über eine E-Mail dazu aufgefordert, seinen Zugang zu aktivieren, in dem er ein Password vergibt. Im Idealfall würde die Zugangsverifikation nur im Heimnetz möglich sein.
\begin{figure}[h]
	\centering
	\includegraphics[scale=0.5]{\imageDir/view-verify-account.JPG}
	\caption{Zugangsaktivierung}
	\label{fig:image-veriy-account}
\end{figure}
\begin{figure}[h]
	\centering
	\includegraphics[scale=0.5]{\imageDir/view-verified-account.JPG}
	\caption{Bestätigung der Aktivierung}
	\label{fig:image-veriied-account}
\end{figure}

\subsubsection{\emph{Client Login}}
Die Abbildung \ref{fig:image-sequence-client-login} zeigt das Sequenzdiagramm, welches den Ablauf des Logins eines registrierten Benutzers über einen mobilen \emph{Client} beschreibt.
\begin{figure}[h]
	\centering
	\includegraphics[scale=0.55]{\imageDir/sequence-client-login.jpg}
	\caption{Sequenzdiagramm des Benutzerlogins}
	\label{fig:image-sequence-client-login}
\end{figure}
\ \newpage
Im Zuge des Benutzerlogins wird der mobile \emph{Client} am \emph{(Auth-Service)}, der die Authentifizierung und Benutzerverwaltung verantwortlich ist und am Applikationsservice \emph{(App-Service)}, der mit der Hardware und dem \emph{Messaging} Dienst interagiert, registriert. Der \emph{Auth-Service} erstellt bei jedem Login einen neuen \emph{OAuth2-Client} und löscht gegebenenfalls einen bereits bestehenden  \emph{OAuth2-Client} für den aktuellen mobilen \emph{Client}. Das Erstellen eines \emph{OAuth2-Clients} für jeden mobilen \emph{Client} wird durchgeführt, da mit der \emph{Client}-Applikation keine \emph{Oauth2-Client} Zugangsdaten an die mobilen \emph{Clients} mit ausgeliefert werden sollen.
\newline
\newline
Dem mobilen \emph{Client} wird bei einem Login ein Authentifizierungstoken für den \emph{Cloud} Dienst übermittelt, mit dem sich der mobile \emph{Client} am \emph{Cloud} Dienst anmelden kann. Nachdem Login eines mobilen \emph{Clients} am \emph{Cloud} Dienst holt sich der mobile \emph{Client} seine eindeutige Id vom \emph{Cloud} Dienst in Form eines Tokens, der am \emph{Auth-Service} registriert wird, der wiederum den Token am \emph{App-Service} registriert, damit dieser in der Lage ist, Nachrichten an die registrierten mobilen \emph{Clients} zu versenden.  

\subsubsection{Sicherheitsverstoß melden}
\begin{figure}[h]
	\centering
	\includegraphics[scale=0.7]{\imageDir/sequence-incident.jpg}
	\caption{Sequenzdiagramm für das Behandelns eines Sicherheitsverstoßes}
	\label{fig:image-sequence-incident}
\end{figure}
\ \newline
Die Abbildung \ref{fig:image-sequence-incident} zeigt das Sequenzdiagramm für das Behandeln eines Sicherheitsverstoßes, der von der Sensorapplikation erkannt und dem Applikationsservice mitgeteilt wird. Der Sicherheitsverstoß wird über den \emph{Cloud} Dienst an die mobilen \emph{Clients} gemeldet, wobei einerseits eine Nachricht an die mobilen \emph{Clients} versendet wird, sowie das gemachte Bild in einer Onlinedatenbank den mobilen \emph{Clients} zum Download zur Verfügung gestellt wird. Die Benutzer können jederzeit auf die Onlinedatenbank zugreifen und sich die Bilder auf ihren jeweiligen mobilen \emph{Clients} herunterladen. 
\newline
\newline
Der Ansatz einen \emph{Cloud} Dienst zu verwenden sorgt dafür, dass das \emph{RPISec} System entlastet wird, da der Datenfluss und die Netzwerkzugriffe vom System ins Internet sowie umgekehrt minimiert werden. Die Daten müssen nur einmalig in die \emph{Cloud} hochgeladen werden und die Benutzer können jederzeit, beliebig oft und von jedem beliebigen mobilen \emph{Client} darauf zugreifen.
\newpage
 
\subsubsection{Nachrichtenempfang am mobilen \emph{Client}}
Die Abbildung \ref{fig:image-sequence-client-notification} zeigt den Ablauf einer Benachrichtigung eines mobilen \emph{Clients} über den \emph{Messaging} Dienst.
\begin{figure}[h]
	\centering
	\includegraphics[scale=0.7]{\imageDir/sequence-client-notification.jpg}
	\caption{Sequenzdiagramm der Benachrichtigung eines mobilen \emph{Clients}}
	\label{fig:image-sequence-client-notification}
\end{figure}
\ \newline
Nachdem das System die mobilen \emph{Clients} via dem \emph{Messaging} Dienst über einen Sicherheitsverstoß benachrichtigt hat, erhalten die mobilen \emph{Client}-Anwendungen die Nachricht von dem \emph{Messaging} Dienst und zeigen diese an. Nachdem die Benutzer auf die Nachricht geklickt haben, wird eine \emph{Activity} für das Anzeigen der Bilder geöffnet, die alle bereits gespeicherten Bilder und das neu geladene Bild anzeigt. Diese Daten werden von der Onlinedatenbank bereitgestellt. 